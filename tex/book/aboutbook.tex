\chapter{About this book}
This book has the form -- structure, layout, composition, design -- of a traditional (\emph{dead tree}) book, but it is not, for different reasons, most of those having to do with automation. For example, editions are much more frequent than usual, and no printed version is planned for the foreseeable future - in addition to a release date, there will also be a version and release number attached to it. There might be different language versions available, and there might even be different titles to adapt to personal preference, for example in the naming of the main subject.  It is a fact that not even the name of the language is shared between its native speakers, and, although it must be clear that I speak the Arubian variant, it must also be clear that I could not care less if the name of the language were to be spelled \emph{Papiamento}, \emph{Papiamentu} or \emph{Papiamen}. The scientific approach is to recogize and adopt all orthographic variants discovered 'in the wild', and only occasionally flag some as deprecated, archaic or erroneous.

Indeed, most of this publication is not even written, but programmed, and this enables some features not found in traditional language publications. Almost all text is 'pulled in' from data storage with programs that are called from the main text composing engine, and reflect the status of the data model and instances of its different type at a specific moment in time. This data model is fluid, and the number of lemmata will increase over time, as are the attributes attached to them. Although this publication, that is intended to resemble something between grammar books, languages course and a lexicon, is the main project deliverable, the data repository will be available in other forms also. For the technically inclined, the colophon will have an up-to-date description of the technology that is used for every edition.


\chapter{Lexicon scope and principles}
\section{Loanwords}
Creole languages have a number of words in the lexicon that are borrowed from other languages, and \pap{} is no exception. For the purposes of composing a lexicon, a dilemma is where to stop including lemmata from other languages. From the code switching point of view of the language user, English, Nederlands and Castiliano can be seen as proper subsets of the \pap{} language; it would not seem a good idea to include all their words in the \pap{} lexicon. So for this publication, the following criteria have been established:
\begin{enumerate}
\item The lemma is part of one of the official, governmental word lists
\item The lemma has a certain frequency in the corpus
\item The lemma is used with significant difference among \pap{} speakers compared to the majority of language users
\item The lemma constitutes a \emph{false friend}
\end{enumerate}
These criteria can overlap, of course.
