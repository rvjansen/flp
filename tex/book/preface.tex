\chapter{Preface}
What is before you now, is the result of an experiment that I set out and documented in 2003 while working as a contractor, designing the Metadata Repository in the Data Modelling Support department of ABN AMRO Head Office, and subsequently discussed with Mario Dijkhoff and Willy Martin. It was to be only ten years later, in the fall of 2013, that I could set aside the time for working on it. In the months September and October the datamodel and code was put together, and an initial, incomplete mapping of its sources was done. Only in the years thereafter the mappings were complete, with the help of several native speakers, mentioned below. It is, in short, the book that I was hoping to find around 1990 when I first was introduced to the \pap{} language, but could not find. The 1990's saw the release of several vocabularies and dictionaries, but the all-encompassing, scientific lexicon is still nowhere to be found. This is an attempt to start the development of one.

Having experienced the detrimental effects of intellectual property rights and contractual obligations to publishers in my first short trip around the lexicographic world, it became clear to me that the work should be completely open and non-encumbered by any claims. For this reason, all information is in the public domain, and even the complete software stack on which it is built, is completely open source. Also, it builds on publicly available data, of which Wordnet is a prime example. The work itself is licenced Creative Commons, NC. It is, and will always stay completely free for anyone who wants to use it, and it is meant to give something back to the Arubian people, and all others that share my love for the \pap{} language.

\section{Thank you}
First and foremost my wife Venetia is thanked here, for being my inspiration and language teacher. She has taught me to speak the language through a severe \emph{natural method} approach, by declining to speak Dutch to me during the first years of our love affair.

Also, my sincere gratitude goes out to:
[tbd].





