\chapter{Remarks on grammar}

\section{Nouns}
The nouns of \pap{} are sourced from a number of languages and can
have a multitude of spelling variants apart from the phonological
versus etymological paradigms. A number of nouns have obtained
different meanings from their source languages; these are flagged in
the main dictionary as such. Also, for Dutch nouns, some contractions
have developed over the course of history, like \emph{flie} and
\emph{spiel}, for resp. \emph{vlieger} and \emph{spiegel} (kite and
mirror).

Several authors \cite[47]{holm1988} have expressed the opinion that \pap{} has been
\emph{relexified} to a Spanish lexicon from earlier Indian, Portuguese
and African forms, and that this process is continuing. Where deemed
appropriate, this has also been flagged in the dictionary.

\subsection{Plural marker}
The plural suffix for every noun is \textbf{-nan}. But: no native
speaker will use a plural marker if the plural is obvious, for example
when a number is in front of the noun. So, the plural of chicken is
\emph{galiñanan}, but we speak about \emph{dos galiña\footnote{where
    the correct etymological spelling should probably be 'galinha'.}}. Only when
plural is indicated without any other clue of the cardinality -nan is
used, and unwarranted use is a giveaway for being a foreigner.
 
\section{Pronouns}

\subsection{Personal pronouns}
\begin{table}[htpb]\caption{Personal pronouns}
\begin{tabularx}{\textwidth}{>{\bfseries}lX}
\toprule
mi & I\\
bo & you \\
e & he / she \\
nos & we \\
boso(nan) & you \\
nan & they \\
\bottomrule
\end{tabularx}
\end{table}
The forms \emph{ami}, \emph{abo}, \emph{anos}, etc., can be used for emphasis, most in the Arubian variant.

\subsection{The second person formal pronoun}
The so-called T/V distinction (like \emph{tu} versus \emph{vous} in French), does exist in \pap{}, but with a slight twist: the informal form is \emph{bo}, but for formal usage, indicating respect or politeness, the third person form is used without using the pronoun. Examples of this usage are:
\begin{enumerate}
\item \emph{Mama kier mas the?} (Does mother want more tea?)
\item \emph{Frans, ta kico Frans kier come awenochi?} (What does Frans want to eat tonight?)
\end{enumerate}
Instead, \emph{Bo} is often used in an impersonal way to indicate some generic phenomenon, like French \emph{on} or Dutch \emph{men}, or where English would used a relative pronoun like in \emph{People who need to leave, should leave now}. Example: \emph{Si b'a bebe muchu, keda leu di bo auto} (``If you had too much to drink, stay away from your car'', in the sense of ``People who drink too much should not drive''). If not known or anticipated, this construct can be quite startling for the person addressed - it is not about you.

\subsection{Dummy pronouns}
The dummy pronoun (``'\emph{It}' is raining'', ``Het regent'') is not used, instead: ``Awa ta yobe" (Aruba) or ``Awa ta cai'' (Curaçao), ``Il fait beau'' becomes ``Tempo ta bon'.

\section{Verbs}
\subsection{The passive construction}
\pap{} has three passive markers: \textbf{wordu} (\textbf{wordo} on Aruba), \textbf{ser} and
\textbf{keda}. The first of these has European roots ( 'worden' is used as the
passive marker in the Dutch language), and the other two are Iberic. The passive construction consists of one of these three
passive verb markers and a past participle. The \textbf{keda} marker
is a \emph{false friend} for Dutch speaking Antilleans; the Dutch sentence
\emph{het portret is mooi gebleven} even has two of these: \emph{portret} in
Dutch is only used for pictures that contain people or only their
faces, but can be used by Antilleans for any photograph, but \emph{is mooi gebleven}, which is a literal translation of \emph{a
keda bunita}, just means 'is nice', while in Dutch this cannot mean
something else than 'something happened to it, but it still is nice',
analogous to the Spanish meaning of \emph{quedar}.

\emph{Wordu} is the oldest form, and \emph{ser} and \emph{keda} are a more recently
introduced, almost 80 years after the first printed occurence of
\emph{wordu} from 1852. All seem to be structural borrowings for a language
that earlier \cite[199]{sanchez2005} possessed a passive form without passive
marker. 

\subsection{-ndo Borrowings}
\emph{-ndo} is an Iberic structural borrowing and appears to only work well on
verbs with Iberic etymological roots. When used on verbs that have
other origins, like in the cases of \emph{wakiendo}, (looking, from
Du. 'waken', to guard (lit. staying awake) and \emph{zwaaimento} (from
Du. zwaaien, to swing) it evokes
strong feelings of discontent and mispreciation. Its Iberic forms are frequent in
'educated' speech, but it should be noted that \pap{} can do without it: there is no reason to say \emph{*mi ta canando bay mi
  cas}, because \emph{mi ta cana bai mi cas} works perfectly well \cite[]{thijsentrimon2013} - also
an example of (probably) african substrate verb accumulation.

Also when an indictor of a process in progress is indispensable, \pap{} does not need
this borrowed conjugation, for example:
\begin{itemize}
\item \emph{Mi tabata feita ora coriente a bai} (I was shaving when we
  lost power)
\end{itemize}
is better use of the language that to put \emph{feitando} in there.


\subsection{Verb tenses}
Verbs are not subject to elaborate conjugation, as all tenses are
constructed through short tense modifiers: \textbf{ta} for present, \textbf{tabata (tawata)}
for progressive and conditional, \textbf{a} for past and \textbf{lo} for future.

\subsubsection{Future tense}
\pap{} has a future tense marker that does not need to be adjacent to the other
TMA\footnote{tense, mood, aspect} markers. \emph{Lo} is used for
events that will take place in the future, with \emph{tabata} as a
conditional marker:
\begin{enumerate}
\item \emph{Nos lo wak mañan ta ken a gana} (tomorrow we will see 
  who has won)
\item \emph{Lo nos scucha mas ora el a haña su cell bek} (we will hear
  more when she gets her cellphone back)
\item \emph{Lo tabata hopi mas mehor si bo no a grita e mucha} (It would have been
  better if you had not scolded the child
\end{enumerate}
\textbf{Lo} can also be put in front of the subject.
For the immediate future \textbf{ta bai} can be used: \emph{Warda un
  rato, mi ta bai traha koffie mesora}.

An in-progress aspect is indicated in the following way:
\begin{itemize}
\item Ora Susy a app mi tabata traha koffie (When Suzy whatsapped I was
  making coffee)
\end{itemize}

\subsection{Verbs that do not take 'ta' in the present tense}
A small number of verbs do not take 'ta' in front. These are:
\begin{enumerate}
\item tin (to have)
\item por (to be able to)
\item ke (to want)
\item sa (to know)
\item conose (to know)
\item mester (must)
\end{enumerate}
The present tense of 'ta' obviously leaves out a second 'ta'.

\subsection{The verb 'tin'}
The verb \textbf{tin} indicates possession or containment.
 \begin{itemize}
\item \emph{Un siman tin shiete dia} (A year has twelve months)
 \item \emph{Coco tin un pushi bunita} (Coco has a nice cat)
 \end{itemize}
 
The past tense of tin is tabatin. The imperative is tene.
 
 \begin{itemize}
 \item \emph{Tene cuidao} (be careful)
 \item \emph{Tene Bonairu limpi} (Keep Bonaire clean)
 \end{itemize}

Sometimes (especially at the beginning of a phrase) tin means 'there
is' or 'there are'.
 
\begin{itemize}
\item \emph{Tin un pushi den e camber} (There is a cat in the bedroom) 
\item \emph{Tin mas berdura?} (Are there any vegetables left?)
\item \emph{Awe no tin lechi} (No Milk Today)
\end{itemize}

\emph{Tin} can not be used in the sense of 'to undergo'  (like in
English "I had the flu" or "We had a great time"). (These can be
expressed as \emph{M'a haña ferkout},
resp.\emph{ Nos a pasa hopi bon}.)
 
 There are other cases where other languages would use a form of 'to
 be' but \pap{} uses 'tin' or 'sinti' (feel), like in the following examples:

\begin{itemize}
\item \emph{Tin hopi biento awe} . It is very windy. (Literally: exists much
  wind today.)
\item \emph{Mi tin hamber}. (I am hungry. (Literally: I have hunger, compare Du. Ik heb honger))
\item \emph{Mi ta sinti kalor}. (I feel warm(th).)
\item \emph{Quanto aña bo tin?} (What is your age? (Literally: how many years you have?))
\end{itemize}
